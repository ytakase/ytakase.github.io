\documentclass[a4j]{jarticle}
\title{hoge}
\begin{document}
\section{実験}
ここまでで提案したシステムを実際の塗装作業を模したタスクへ適応し,利用者のタスクへの熟達の
違いが計測できるかどうか確認するため次のような計測を行った.
\subsection{実験目的}
作成したシステムの評価として作業の熟練度による動作の傾向の差を測定する.
なぜなら,うんぬんかんぬん

被験者は,熟練度の高い人のサンプルとして,塗装学校で講師を務めるA 人,熟練度の低いものとしてXXの人を
B 人集めた.被験者は全員右利きの男性である
\subsection{評価項目の作成}
評価項目では,事前にXXへの聞き取り調査を行い,塗装作業において熟練者と非熟練者間での違いの現れやすい項目を
列挙した.表\ref{table:1}に示す
\begin{table}[t]
    \caption{ほげほげ}
    \label{table:1}
    \begin{tabular}{ccc}
        質問項目                                     & 回答                                                 & 対応する計測項目     \\
        \hline\hline
        下塗りと中塗りで,動作に違いはあるか         & 動作に大きな違いはない                               & ローラーブラシの速度 \\
        ローラーブラシ塗装において肝となる部分は何か & $1 m^2$ あたりの塗料を塗布する量,均一に塗れているか & 塗装面の凹凸
    \end{tabular}
\end{table}
\subsection{計測タスク}
計測のタスクはうんぬんかんぬん
\subsection{実験環境}
\subsection{実験結果}
計測結果について以下に述べる

\section{議論}
\end{document}